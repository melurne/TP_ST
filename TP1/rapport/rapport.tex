\documentclass[oneside,a4paper,12pt]{article}
\usepackage{graphicx}
\usepackage{amsmath}
\usepackage{listings}
\graphicspath{{~/templates/}, {../images/}}

\makeindex
\begin{document}
	\begin{titlepage}
		\includegraphics[width=4cm]{logopopo.png}
		\hspace*{\fill}
		\includegraphics[width=6cm]{logouniv.png}
		
		\begin{center}
			\vspace{1cm}
			\textbf{TP Support de Transmission}\\
			\textbf{Oscillateur microonde}\\
			\vspace{1cm}
			\textbf{Maxence LAURENT, Thibault VOLLERIN, Maxence NEUS}\\
			\vspace{3cm}
			%\includegraphics[width=13cm]{titlepage.png}\\
			\vspace{\fill}
			\textbf{Mars 2022}\\
		\end{center}
	\end{titlepage}
	
	\tableofcontents
	
	\vspace{5cm}
	
	\begin{abstract}
	
	\end{abstract}

	\newpage

	\section{Préparation}
	
	

	\newpage

	\section{Manipulations}

	\subsection*{Paramètres du montage}
	$ f_{0} = 6 GHz $.\\


	$ P_{Wattmetre} = 4.2 dBm $ soit\\
	$ P_{0} = P_{Wattmetre} + 0.6 dBm + 10 dBm = 14.8 dBm $ \\


	$ P_{Analyseur de spectre} = 12.6 dBm $ soit \\
	$ P_{0} = P_{Analyseur de spectre} + 2* 0.6 dBm = 13.8 dBm $\\


	$ P_{0 Analyseur de spectre} < P_{0 Wattmetre} $ Car l'analyseur de spectre mesure la puissance à une seule fréquence, alors que le Wattmetre mesure la puissance totale du signal.\\

	$ \eta = \frac{P_{out}}{P_{in}} $
	\[ P_{in} = U*I = 15*0.46 = 6.9mW \]
	On prendra $P_{out} = P_{0 Analyseur de spectre} $ pour avoir le rendement à $ f_{0} $.

	\[ \eta = \frac{10^{\frac{13.8}{10}}}{6.9} = 0.29 \]

	\subsection{Pureté spectrale}

	\paragraph*{Bruit de modulation de phase}

	pic central $ f_{0} = 6.000105 GHz P = 13.8 dBm $\\
	\[ f = f_{0} + 2KHz P = -13.41 dBm \]
	\[ f = f_{0} + 5KHz P = -40.17 dBm \]
	\[ f = f_{0} + 10KHz P = -46 dBm \]

	On calcule $ \frac{\frac{P_{B}}{P_{0}}}{\Delta f} $ pour nos valeurs de $ f_{m} $
	\begin{itemize}
		\item $ f_{m} = 2kHz \rightarrow \frac{\frac{P_{B}}{P_{0}}}{\Delta f} = -0.027 dBc/Hz $ 
		\item $ f_{m} = 5kHz \rightarrow \frac{\frac{P_{B}}{P_{0}}}{\Delta f} = -0.053 dBc/Hz $
		\item $ f_{m} = 10KHz \rightarrow \frac{\frac{P_{B}}{P_{0}}}{\Delta f} = -0.059 dBc/Hz $
	\end{itemize}

	\paragraph*{Distortion harmonique}

	On peut lire sur l'analyseur de spectre les deux premières armoniques:
	\begin{itemize}
		\item $ n = 2 \rightarrow -25.4 dBm $
		\item $ n = 3 \rightarrow -11.1 dBm $
	\end{itemize}
	\[ D_{h} = \frac{\sum_{n = 2}^{\infty} P_{n}}{P_{0}} = 0.3 \% \]

	\subsection{Stabilité de l'oscillateur}

	\paragraph*{Facteur de pulling} \paragraph{}

	\[ f_{min} = 5.999483 GHz; f_{max} = 5.999898 \]
	donc $ \Delta f_{0} = 415 kHz $

	\paragraph*{Synchronisation d'un oscillateur}\paragraph*{}

	\[ f_{s_{min}} = 6.000364 GHz; f_{s_{max}} = 6.000560 GHz \]
	donc 
	\[ \Delta f_{s} = 196 kHz \]

	avec 
	\[ f_{0} = 6.000537 GHz, P_{0} = 12.6 dBm, P_{gene} = 2.6 dBm \]

	Donc on peut calculer $ Q_{ext} = 1938 $:


	\newpage
	\section{Conclusion}
			

\end{document}
